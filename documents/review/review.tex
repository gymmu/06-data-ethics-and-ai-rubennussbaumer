\documentclass{article}

\usepackage[ngerman]{babel}
\usepackage[utf8]{inputenc}
\usepackage[T1]{fontenc}
\usepackage{hyperref}
\usepackage{csquotes}


\usepackage[
    backend=biber,
    style=apa,
    sortlocale=de_DE,
    natbib=true,
    url=false,
    doi=false,
    sortcites=true,
    sorting=nyt,
    isbn=false,
    hyperref=true,
    backref=false,
    giveninits=false,
    eprint=false]{biblatex}
\addbibresource{../references/bibliography.bib}

\title{Review des Papers "Ethik im Umgang mit Daten" von Nikolay Voropayev}
\author{Ruben Nussbaumer}
\date{\today}

\begin{document}
\maketitle

\abstract{
    Dies ist ein Review der Arbeit zum Thema Ethik im Umgang mit Daten von Nikolay Voropayev.
}

\newpage

\section{Korrekturen}

Da Nikolay sehr viel von KI versteht und tatsächlich gute Quellen verwendet hat, habe ich an dieser Arbeit nicht so viel von Inhaltlicher Korrekheit zu korrigieren. Dennoch habe ich ein paar Punkte herausgesucht die ich korrigiert habe.

Im ganzen Dokument sind überall Rechtschreibefehler und Grammatikfehler. 
Ich empfehle den text einmal durch ein Rechtschreibeprogramm laufen zu lassen, da es zu viele sind um sie hier aufzuzählen.

1.1.1 hätte ich nicht verstanden, ohne das Video zu schauen, was heisst man muss schon grosses Vorwissen haben über KI um das Paper zu verstehen, aber da die Personen, welche das Paper lesen wenig bis gar keine Ahnung haben von KI, hätte ich es versucht einfacher zu beschreiben und das Video wegzumachen.

1.2 und 1.2.1 gehört meiner Meinung nicht in die Einleitung bei einem Paper über KI, Daten und Ethik. Man sollte es am Rande erwähnen, aber nicht eine ganze Seite über Datenschutz schreiben und es dann nicht einmal mit KI in Verbindung zu bringen an erster Stelle.

1.2 und 1.21 wiederholt sich in den Titeln und im Text mehrmals.

In 2.1 wird von einer Lüge erzählt, welche die grossen Tech Firmen immer erzählen sollen. Es ist eine Missinformation, weil Nikolay nicht wirklich weiss ob es eine Lüge ist, denn er hat das nicht mit ordentlichen Quellen hinterlegt und Beweise aufgezeigt. Also hat er eigentlich eine einfache Behauptung aufgestellt ohne etwas zu wissen und da ist ein Paper definitiv nicht der richtige Ort dafür.

\section{Verbesserungen}
Ich finde das Paper sehr gut geschrieben trotzdem habe ich einige kleine Verbesserungen.
Bei der Einleitung den Fokus nicht auf Datenschutz setzen, denn das ist niht das was man wissen muss in der Einleitung.

Mir hat ein bisschen gefehlt, was Ethik überhaupt ist. Man hätte das in der Einleitung noch Klassifizieren können, damit man Nikolays Definition sehen kann und somit bessere Schlüsse ziehen kann, wenn man die Arbeit liest.

\section{Kommentare}

Mir haben die umfangreichen Beispiele gefallen. So hat nicht alles so sehr theoretisch gewirkt und man konnte einen Zusammenhang zur Realität bekommen.
Die Verbesserungsvorschläge in 3 wahren zum Teil auch sehr innovativ und gut beschrieben.
\end{document}
