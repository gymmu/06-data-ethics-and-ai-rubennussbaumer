\documentclass{article}

\usepackage[ngerman]{babel}
\usepackage[utf8]{inputenc}
\usepackage[T1]{fontenc}
\usepackage{hyperref}
\usepackage{csquotes}

\usepackage[
    backend=biber,
    style=apa,
    sortlocale=de_DE,
    natbib=true,
    url=false,
    doi=false,
    sortcites=true,
    sorting=nyt,
    isbn=false,
    hyperref=true,
    backref=false,
    giveninits=false,
    eprint=false]{biblatex}
\addbibresource{../references/bibliography.bib}

\title{Notizen zum Projekt Data Ethics}
\author{Ruben Nussbaumer}
\date{\today}

\begin{document}
\maketitle

\abstract{
    Dieses Dokument ist eine Sammlung von Notizen zu dem Projekt. Die Struktur innerhalb des
    Projektes ist gleich ausgelegt wie in der Hauptarbeit, somit kann hier einfach geschrieben
    werden, und die Teile die man verwenden möchte, kann man direkt in die Hauptdatei ziehen.
}

\tableofcontents

\section{Notizen, KI Funktion}
KI beinhaltet eine grosse Zahl an Technologien, die das Ziel hat Maschinen und Computern die Fähigkeit
zu geben, Aufgaben, welche Menschen Zeit braucht, selbstständig zu erledigen, was normalerweise eine Menschliche Intelligenz benötigt.
Die Grundlegenden Konzepte, wie KI mit diesen Technologien beschaffen ist, sind in den folgenden Abschnitten beschrieben.

1. Die Grundlagen, die KI braucht um überhaupt zu funktionieren, sind Daten. KI bezieht ihr Wissen und die Infos, die sie braucht aus Daten, 
die ihr eingeschleust wurden.

2. KI lernt viel durch Maschinelles Lernen. Sie haben sicher schon einmal gesehen, wie eine KI in einem Spiel platziert wird und diese solange
das Spiel zu verstehen lernt, bis es das ganze Spiel spielen kann. z.B. bei Super Mario.

3. Wie funktionieren Neuronale Netzwerke? Was sind Neuronale Netzwerke? Was sind Schichten?

4. Was ist Deep Learning? In welchem Zusammenhang steht es zu Neuronalen Netzwerken.

5. Anwendungen der KI: 

    -KI kann dabei helfen, die Fehlereffekte zu reduzieren.
    -Spracherkennung
    -Bildverarbeitung
    -Automatisierung
    -Empfehlungen auf persöhnlichen Interessen basierend
    -Spieleintelligenz

 \section{Notizen, KI und Ethik}

 1.Was hat KI mit Ethik zu tun? 
 2.Wie kann KI ethisch helfen: Medizin, Recht, Verhandlungen, etc.
 3.Wie kann KI ethisch gesehen Schaden: Krieg, Spionage, etc.
 4.Wie verhindert man unethische Taten der KI? Beibringen von ethischen Verhaltensweisen, und regulierung der Benutzer.
 5.Wie wird KI unethisch heutzutage verwendet? 
 6.Fazit/Eigene Meinung: KI kann sehr gut sein aber auch sehr schlecht. Abhängig von den Händen in der Sie ist.
 Regulierung der Zugänglichkeit/ Lizenz usw.


\input{section_ai.tex}

\printbibliography

\end{document}
