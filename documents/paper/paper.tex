\documentclass{report}

\usepackage[ngerman]{babel}
\usepackage[utf8]{inputenc}
\usepackage[T1]{fontenc}
\usepackage{hyperref}
\usepackage{csquotes}
\usepackage[a4paper]{geometry}

\usepackage[
    backend=biber,
    style=apa,
    sortlocale=de_DE,
    natbib=true,
    url=false,
    doi=false,
    sortcites=true,
    sorting=nyt,
    isbn=false,
    hyperref=true,
    backref=false,
    giveninits=false,
    eprint=false]{biblatex}
\addbibresource{../references/bibliography.bib}


\title{Ethik im Umgang mit Daten} 
\author{Ruben Nussbaumer}
\date{\today}


\begin{document}

\maketitle

\abstract{
    In diesem Dokument zeige ich auf, wie KI funktioniert und was KI überhaupt ist. 
    Danach erkläre ich und zeige ich auf, wie KI im Zusammenhang mit Ethik steht und habe noch einige Fragen zu diesem Thema beantwortet.
}

\tableofcontents

\chapter{Einleitung }
Haben Sie sich schon einmal gefragt, wie KI funktioniert und ob KI auch eine Art Persöhnlichkeit haben kann und somit auch Entscheidungen treffen, die normalerweise Gefühle brauchen.
Ob und wie KI mit Ethischen Entscheidungen funktioniert und welche positiven und negativen Auswirkungen das haben kann, beantworte ich in diesem Paper.
Da das selber meine erste Auseinandersetzung mit KI und Ethik ist, habe ich selbst sehr viel gelernt, bei dieser Arbeit.
Diese Arbeit dient dazu, dass man sich einen Überblick machen über dieses Thema machen kann und danach kann man sich selbst vertiefen, wenn man möchte.

\input{chap_methode.tex}

\section{Etwas mit Quellen}

Etwas mit Änderung hier am Ende.

Wenn ich eine Quelle zitieren möchte, kann ich das ganze einfach am Ende des Satzes machen \citep{example}. Oder wie \citet{example} sagt, auch mitten im Text.

\printbibliography

\end{document}
